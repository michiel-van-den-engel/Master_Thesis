\section{Conclusion}
\label{seq:conclusion}
In this thesis, we used two different data sets which predictably resulted in different results. This difference is a reminder that there is never one universal best model for every type of data and that no one should assume that one model performs better than the other one because it did last time. We made a total of ten different models with three machine learning models, 6 statistical methods and 1 ensemble method. We conclude that for our data sets some models did outperform others but we did not find any proof that machine learning models outperform statistical methods and vice versa.\\

Our choice for the German data set would be the regression model as it is simple, well explainable and boasts very good quality measures when looking at the RMSE or the MAE. For the Dutch data set the model of choice would be the ARIMAX model as it provides the best results on the 7 and 28 steps ahead horizon while not requiring that much computing power.\\

Another answer we seek to answer is which model performs best per horizon and there do not appear to be large differences per horizon for every model, but for some models this is a large factor. This might have something to do with the large availability of explanatory data which makes the model less dependent on temporal features as we see the performance of the ARIMA and Local Level model decrease a lot with time. This could have something to do with the fact that these models both do not use any of the explanatory data to come to their forecasts. Some future research is needed to make more comparisons between different and even more models in different contexts on multi step ahead forecasts to see if these models perform similarly in other contexts and to include even more models in the comparison for a more complete picture of the available models. On model in particular to take a look on in future research is the SARIMAX model, as this adds a seasonal component to the ARIMAX model what could improve the performance of the model. Another point for future research is not to compare models over different horizons, but study if the number of steps forecast Granger causes the performance of the model more closely with.