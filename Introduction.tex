\section{Introduction}
For a multi-step ahead forecast, there are many different models that a data scientist might use. Many models have different advantages and disadvantages and might be better or worse for a specific situation. Most forecasting models can be classified in one of two different categories: machine learning models and statistical models. In this thesis we will look at these models and research if they perform differently on multi step ahead forecasts.\\

Time series models have a lot of applications, from the prediction of sales \citep{Pavlyshenko2019Machine-LearningForecasting} to proving that the climate crisis is caused by humans \citep{Stern2013AnthropogenicChange}. For this wide range of applications it is important to research the performance of the methods used by researchers. For example \cite{Crane-Droesch2018MachineAgriculture} uses forecasting methods to predict the impact of climate change to the agricultural output from the US midwest. In this thesis we will focus on the application of multi-step forecasting. The models we use can be broken up in two categories: the statistical methods and the machine learning methods. We compare models from these two categories on 7 step (one week), 28 step (four weeks) and 49 step (seven weeks) horizons and we will research if there is a difference in performance of these models on these horizons.\\

One of the studies that compare multiple machine learning models and statistical is done by \cite{Papacharalampous2019ComparisonProcesses} who compare different statistical models and machine learning models for hydrological processes. \cite{Papacharalampous2019ComparisonProcesses} find that for hydrological processes the machine learning models and statistical models do not differ much in performance. In this thesis we will expand the existing literature by adding results for a completely different process. \cite{Alon2001ForecastingMethods} compare an Artificial Neural Network to a more traditional ARIMA model on multi-step forecasting performance and concludes that the Machine Learning Model has a better performance. In this thesis we do this comparison not only for the ARIMA model and the Neural Network, but also for some expanded statistical methods, a tree structured machine learning model and for models that use a multivariate breakdown of the data.\\

To add to the literature we will train ten different models and then test with the Diebold-Mariano test if the forecasts produced by these models are significantly statistically different. The statistical models we will train and compare are the local level model, the regression model, the ARIMA model, the ARIMAX model, the VARIMAX model, the fused lasso model, and the machine learning models are the XGBoost model, a Neural Network and a Multivariate Neural Network. Lastly, we will also create a mixed model which is an aggregate of two 2 statistical and two machine learning models. With these models, we do not find any evidence that the machine learning models perform significantly better than the statistical methods and we also find that for the two different data sets used, there is no one method that performs the best on both the data sets.\\

To test these models we will make use of data from the Picnic customer service department. Picnic is an online supermarket startup which operates in the Netherlands, Germany and France.  With this data, we make a forecast for each of the 10 proposed models and determine which will conclude that the method which predicts the total number of cases best is the ARIMAX model.\\

We will compare the ten individual models We will give a general description of how all of these models work in \autoref{seq:methodology}. We will describe what data we will use for the model estimations to give an understanding of the context in which this comparison is done in \autoref{seq:the data}. We will report the forecast results and estimated parameters of all statistical models and we will perform statistical tests to check if the difference in performance is significant which are described in \autoref{seq:results}. Finally, in \autoref{seq:conclusion} the findings of this thesis are described.

%Most large companies employ a customer service team. These teams solve any issues that customers might have. An agent at a customer success team answers messages coming from customers and helps the customers with issues they might have. If an issue comes in, the customer success systems create a case and the case will be handled by an agent. Cases can come in via different ways, like WhatsApp, Phone, Feedback in the app, Mail and more. If there are too few agents at work, the waiting times will increase and customers will leave the queue. \citep{Zohar2002AdaptiveSupport}\\

%The negative effect of queues is not only seen at customer service operations, but also in restaurants \citep{DeVries2018WorthRevenue}, stores \citep{Lu2013MeasuringPurchases} and physical help desks \citep{Zhou2003LookingBehind}. \cite{Lu2013MeasuringPurchases} concludes that customers are heterogeneous in their sensitivity to queues. The effects of too long queues cannot easily be understated as \cite{Bielen2007WaitingServices} find that the level of satisfaction with the customer service experience impacts the level of customer loyalty. \cite{Davis1998HowSatisfaction} show that the impact of the actual waiting time on customer satisfaction has an increasing influence as the waiting time increases.\\

%At every customer service operation there is a balance to achieve. As there are more agents working the waiting time for customers will decrease, but this costs the company a lot of money \citep{Liu2017InfluenceCenters} and if there are too many agents working, they might not have any cases left and cannot work effectively. To make sure the waiting times keep to a minimal while the resources of the company are not wasted a reliable forecast is needed.\\

%Therefore we want to make a forecast for the number of incoming cases in the customer success operations at the customer service of the Picnic supermarket. We want to look at different models and see what model performs the best. We will compare Machine learning and statistical methods of forecasting and report the quality measures of these methods.\\

%Making a forecast for a customer service operation is notoriously challenging as the number of incoming cases is highly volatile \citep{Hafeez2013PlanningLevels}. The number of customers might increase due to incidents that cannot be taken into account in a forecast model. A situation like this at Picnic in August 2022 there was an issue with dry ice which had as a result that customers could not order any frozen products resulting in an increase of the cases per delivery.\\

%The business at the moment has three main stakeholders, the recruitment team, the long term planner and the short term planner. From past experience however, we learn that if a forecast is available, more applications of the forecast can be discovered. The three main use cases that would start using the forecast immediately are:
%\begin{enumerate}
%    \item The recruitment for customer service team needs to make a correct assessment about the number of agents in service and how many agents should be hired. This information is needed months into the future for a stable hiring program as new agents need to be found and trained.
%    \item A customer service planner is also benefited by this forecast as the planner needs to make sure there are enough agents but for this purpose it is required to make a forecast that goes far into the future making it less reliable.
%    \item Each week, the planner might look again at the planning for next week. As forecasts try to predict the future based on lagged data, the forecast will improve when the time frame moves closer to reality. There might be a shortage of planned agents due to agents quitting and or forecast adjustments and the operations might want to ask agents to work extra and possibly add extra benefits for agents to do extra work.
%\end{enumerate}
%As described above there are multiple use cases for this forecast. This results in some problems with making the forecast. The forecast is used at the intervals of 1 week, 4 weeks and 7 week intervals. These used cases might benefit from different trained models and in this Thesis we will compare multiple models on different time intervals and see if a time varying combination of these models might perform better over different time periods.